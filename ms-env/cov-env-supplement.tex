%COVID-19 R0 and environment - SUPPLEMENT

\documentclass[12pt,english,a4paper]{article}
\usepackage[utf8]{inputenc}
\usepackage{babel}
\markright{Pearse et al.\hfill Building up biogeography: pattern to process\hfill}
\usepackage{geometry}
\geometry{verbose,a4paper,tmargin=2.54cm,bmargin=2.54cm,lmargin=2.54cm,rmargin=2.54cm}
%\geometry{verbose,letterpaper,tmargin=.1cm,bmargin=.1cm,lmargin=.1cm,rmargin=.1cm}
\usepackage{graphicx}
\DeclareGraphicsExtensions{.pdf,.png,.jpg}
\usepackage{amssymb,amsmath}
\usepackage{epstopdf}
\usepackage{supertabular}
\DeclareGraphicsRule{.tif}{png}{.png}{`convert #1 `dirname #1`/`basename #1 .tif`.png}
\usepackage{url}
\usepackage{subcaption}
\usepackage[super]{nth}
\usepackage{lineno} \linenumbers
\usepackage[doublespacing]{setspace}
\usepackage[parfill]{parskip}
\setlength{\parindent}{0pt}
\usepackage[super,sort&compress,comma]{natbib}
\usepackage{subcaption} % subfigures
% breaking urls so they don't run past the margin in bibliography:
\usepackage{url}
\def\UrlBreaks{\do\/\do-}
\usepackage{breakurl}

%%%%%% Keep this block in all manuscripts to allow commenting %%%%%%
\usepackage{color}
\usepackage{setspace}
\usepackage{soul} %to get highlights with todo notes
\usepackage[backgroundcolor=yellow,textsize=tiny,textwidth=2.17cm]{todonotes}
\newcounter{todocounter}
\newcommand{\todonum}[2][]{
\stepcounter{todocounter}\todo[#1]{\thetodocounter: #2}
% \setstretch{0.8}% spacing
}
\newcommand{\hlfix}[2]{\texthl{#1}\todonum{#2}~} %this needs the soul package
% Shortcuts
\usepackage{xspace}
\newcommand{\virus}{SARS-CoV-2\xspace}
\newcommand{\disease}{COVID-19\xspace}
\newcommand{\RO}{$R_0$\xspace}

\usepackage[final]{changes}
\setdeletedmarkup{\textcolor{red}{\sout{#1}}}

\renewcommand{\thetable}{S\arabic{table}} % These for supplementary tables/figs
\renewcommand{\thefigure}{S\arabic{figure}}


\begin{document}
\setlength{\parindent}{0pt}

\section*{}

Supporting Information for: \\ Environment influences SARS-CoV-2 transmission in the absence of non-pharmaceutical interventions


\textbf{Authors:} Thomas P. Smith$^{1,*}$,
% Contributors alphabetised
Seth Flaxman$^2$, Amanda S. Gallinat$^3$, Sylvia P. Kinosian$^3$, Michael Stemkovski$^3$,
H. Juliette T. Unwin$^4$,
Oliver J. Watson$^4$, Charles Whittaker$^4$,
% Grant team alphabetised
Lorenzo Cattarino$^4$, Ilaria Dorigatti$^4$, Michael Tristem$^1$,
William D.\ Pearse$^{1,3*}$

$^1$ Department of Life Sciences, Imperial College London, Silwood Park, Ascot, Berkshire, \
 SL5 7PY, UK
 
$^2$ Department of Mathematics, Imperial College London, London SW7 2AZ.

$^3$ Department of Biology \& Ecology Center, Utah State University,
5305 Old Main Hill, Logan UT, 84322

$^4$ MRC Centre for Global Infectious Disease Analysis, Imperial College London, \ 
Norfolk Place, London, W2 1PG, UK 

$^*$To whom correspondence should be addressed:
\texttt{thomas.smith1@imperial.ac.uk} and \\
\texttt{will.pearse@imperial.ac.uk}

\clearpage

\tableofcontents

\clearpage

\section{Collinearity of Climate Variables}

As stated in the main text, correlations between the climate variables (temperature, absolute humidity, UV radiation levels) mean that we could not disentangle each variable's contributions to \RO in multiple regression analysis. This can be seen in the high variance inflation factors for all climate variables when these predictors are regressed together (table \ref{tab:full_regression}).

\begin{table}[ht]
\centering
\caption{\textbf{Multiple regression with all climate variables results in high inflation of variance.} Multiple $r^2 = 0.5951$, \textsl{adjusted} $r^2 = 0.5524$, $F_{4,38} = 13.96$, $p < 0.001$. Scaled estimates are coefficients when predictors are scaled to have mean $= 0$ and SD $= 1$. The variance of environmental parameters are inflated due to their strong collinearity (VIF = variance inflation factor). * = $p < 0.05$}
\begin{tabular}{rrrrrr}
  \hline
 & Scaled Estimate & Std. Error &  $t$ value & $p$-value & VIF \\ 
  \hline
  (Intercept) & 2.5600 & 0.0496 & 51.59 & $<0.001$* & - \\ 
  Temperature & -0.3925 & 0.1292 & -3.04 & 0.0045* & 6.608 \\ 
  Abs. Humidity & 0.0863 & 0.1157 & 0.75 & 0.4604 & 5.299 \\ 
  UV radiation & 0.0982 & 0.0753 & 1.30 & 0.2006 & 2.242 \\ 
  $\log_{10}$(Pop density) & 0.2229 & 0.0570 & 3.91 & $<0.001$* & 1.285  \\ 
   \hline
\end{tabular}
\label{tab:full_regression}
\end{table}

To further illustrate this, we perform a principal components analysis (PCA) on the three climate variables in our dataset, plus population density. Eigenvalues of the first two components of the combined datasets together accounted for 83\% of the total variance. Temperature and absolute humidity fall mainly along the axis of PC1, which accounts for 55\% of the variation. This close relationship is expected given that absolute humidity is a function of temperature (see main text eq. 1). UV radiation is correlated with temperature and humidity, but less so, and contributes more to PC2 which also comprises population density and explains 28\% of the variance (fig. \ref{fig:pca}).

\begin{figure}[ht]
  \centering
  \includegraphics[width=.5\linewidth]{figures/pca_plot_USA.png}  
  \caption{ {\bf Results of PCA (PC1 vs PC2) carried out on climate variables and population density for each state.} Arrows correspond to each of the analysis variables. The proportion of variance explained by each axis is shown on the axis labels. The climate variables condense mainly into PC1, whilst population density explains more of the variance in PC2.}
\label{fig:pca}
\end{figure}


For further analysis we therefore chose the best fitting climate variable, assessed by Pearson's $r$ (table \ref{tab:climate_selection}). Based on this, we used temperature as our climate variable in further analyses.

\begin{table}[ht]
\centering
\caption{\textbf{Assessment of the best fitting climate variable.} Comparison of correlation coefficients between \RO and each of the climate variables; temperature, absolute humidity and surface UV levels. Temperature shows the highest correlation with \RO.}
\begin{tabular}{lr}
  \hline
  Variable & Pearson's $r$  \\ 
  \hline
  Temperature & -0.675 \\
  Absolute humidity & -0.561 \\
  UV & -0.357 \\
  \hline
\end{tabular}
\label{tab:climate_selection}
\end{table}


\section{Effects of environment during lockdown}

When regressing lockdown $R_t$ (mean $R_t$ across the two weeks following a state-wide stay-at-home mandate) against temperature and log$_{10}$-transformed population density, we find no significant effect of temperature, and a much lower coefficient for population density than in the \RO regression model (table \ref{tab:lockdown_model}).

\begin{table}[ht]
\centering
\caption{{\bf Lockdown reduces the effects of environment on $R$.} Multiple $r^2 = 0.2197$, \textsl{adjusted} $r^2 = 0.1775$, $F_{2,37} = 5.209$, $p = 0.0102$. Scaled estimates are coefficients when predictors are scaled to have mean $= 0$ and SD $= 1$}
\begin{tabular}{rrrrr}
  \hline
  & Scaled Estimate & Std. Error & t value & p-value \\ 
  \hline
  Intercept & 1.1253 & 0.0374 & 30.12 & $< 0.001$* \\ 
  Temperature & -0.0017 & 0.0392 & -0.04 & 0.9665 \\ 
  $\log_{10}$(Pop density) & 0.1217 & 0.0392 & 3.10 & 0.0037* \\ 
   \hline
\end{tabular}
\label{tab:lockdown_model}
\end{table}


\section{Lockdown as an interaction term}

Here we combine the pre-lockdown \RO and during-lockdown $R_t$ data for the USA and perform multiple linear regression with lockdown as an additional parameter, to further investigate the impact of mobility restrictions (\emph{i.e.} lockdown) compared to the environment. We restrict the environmental parameters to temperature and population density due to the collinearity between temperature, UV radiation and humidity (see section 1). We first incorporate lockdown as a binary (\emph{i.e.} in lockdown, or not in lockdown) additive effect and test the effects on $R$:

\begin{equation*}
    R \sim Temperature + \log_{10}(Population\ density) + Lockdown
\end{equation*}

As expected, we find that lockdown has a significant effect on $R$, with an order of magnitude greater strength than either environmental parameter (table \ref{tab:additive_model}).

% additive
\begin{table}[ht]
\centering
\caption{{\bf Lockdown has a stronger effect on $R$ than environment} Multiple $r^2 = 0.8637$, \textsl{adjusted} $r^2 = 0.8583$, $F_{3,76} = 160.5$, $p < 0.001$. Scaled estimates are coefficients when predictors are scaled to have mean $= 0$ and SD $= 1$}
\begin{tabular}{rrrrr}
  \hline
 & Scaled Estimate & Std. Error & t value & p-value \\ 
  \hline
 (Intercept) & 2.47351 & 0.05279 & 46.854 & $< 0.001$* \\ 
  Temperature & -0.17529 & 0.04124 & -4.251 & $< 0.001$* \\ 
  $\log_{10}$(Pop density) & 0.15151 & 0.03580 & 4.232 & $< 0.001$* \\ 
  Lockdown & -1.26176 & 0.08001 & -15.770 & $< 0.001$* \\ 
   \hline
\end{tabular}
\label{tab:additive_model}
\end{table}

Subsequently we ask whether lockdown mediates the environmental effects by incorporating this as an interaction term, \emph{i.e.}:

\begin{equation*}
\begin{aligned}
    R \sim{} & Temperature + \log_{10}(Population\ density) + Lockdown + \\
             &  Temperature:Lockdown + \log_{10}(Population\ density):Lockdown 
\end{aligned}
\end{equation*}

\begin{table}[ht]
\centering
\caption{{\bf Lockdown conditions mediate the effects of environment on $R$.} Multiple $r^2 = 0.8927$, \textsl{adjusted} $r^2 = 0.8854$, $F_{5,74} = 123.1$, $p < 0.001$. Scaled estimates are coefficients when predictors are scaled to have mean $= 0$ and SD $= 1$}
\begin{tabular}{rrrrr}
  \hline
 & Scaled Estimate & Std. Error & t value & p-value \\ 
  \hline
  (Intercept) & 2.41331 & 0.04992 & 48.346 & $< 0.001$* \\ 
  Temperature & -0.29734 & 0.04852 & -6.128 & $< 0.001$* \\ 
  $\log_{10}$(Pop density) & 0.19084 & 0.04541 & 4.202 & $< 0.001$* \\ 
  Lockdown & -1.28701 & 0.07222 & -17.822 & $< 0.001$* \\ 
  Temperature:Lockdown & 0.29527 & 0.07524 & 3.924 & $< 0.001$* \\ 
  $\log_{10}$(Pop density):Lockdown & -0.06995 & 0.06442 & -1.086 & 0.281  \\ 
   \hline
\end{tabular}
\label{tab:interaction_model}
\end{table}

Here we find that there is a significant interaction between lockdown and temperature, but not population density (table \ref{tab:interaction_model}), \emph{i.e.} the effects of temperature on $R$ are dependant upon lockdown status. This interaction model is preferred over the purely additive model (ANOVA, $F_{2,74} = 9.996$, $p < 0.001$).



\section{Effect of environment on recreational mobility}

Mobility trends for parks were omitted from estimates of $R_t$ both from the datasets that we used our external validation exercise, as well as our own Bayesian modelling (as significant contact events are assumed to be negligible). However, here we further investigate the impacts of environmental temperature on outdoor recreational mobility levels as a potential confounding factor in these types of analysis (\emph{i.e.} ``do people go to the park more when it's warm?''). 
We first detrend the data using the diff() function in R, as the effects of policy are expected to have a large impact on the realised mobility levels. This allows us to compare whether an increase in environmental temperature at the daily level causes a similar increase in recreational mobility. When we compare daily changes in mobility against daily changes in temperature across all US states, between February and June 2020, we find no significant correlation ($t_{5276} = -0.64628$, $r = -0.0089$, $p = 0.518$). 



\section{Bayesian model fit}

Here we provide the full Bayesian model coefficients and results of posterior predictive checks to ensure that all chains had mixed and converged (table \ref{tab:posterior_summary}).

\begin{table}[ht]
\centering
\tiny
\begin{tabular}{rrrrrrrrrrr}
  \hline
 & mean & se\_mean & sd & 2.5\% & 25\% & 50\% & 75\% & 97.5\% & n\_eff & $\hat{r}$ \\ 
  \hline
  $\alpha_1$ & 2.14 & 0.01 & 0.45 & 1.21 & 1.85 & 2.16 & 2.46 & 2.98 & 1394.56 & 1.00 \\ 
  $\alpha_2$ & 0.01 & 0.01 & 0.27 & -0.51 & -0.17 & 0.00 & 0.19 & 0.55 & 2401.50 & 1.00 \\ 
  $\alpha_3$ & 0.89 & 0.01 & 0.50 & -0.10 & 0.55 & 0.90 & 1.23 & 1.87 & 3778.68 & 1.00 \\ 
  $\alpha_1^{state}$ & -0.00 & 0.00 & 0.31 & -0.67 & -0.11 & -0.00 & 0.11 & 0.66 & 11062.97 & 1.00 \\ 
  $\alpha_2^{state}$ & 0.00 & 0.00 & 0.29 & -0.63 & -0.11 & 0.00 & 0.11 & 0.66 & 13676.39 & 1.00 \\ 
  $\alpha_3^{state}$ & 0.00 & 0.00 & 0.30 & -0.65 & -0.11 & 0.00 & 0.12 & 0.65 & 12106.48 & 1.00 \\ 
  $\alpha_4^{state}$ & 0.00 & 0.00 & 0.30 & -0.64 & -0.10 & 0.00 & 0.11 & 0.65 & 12138.84 & 1.00 \\ 
  $\alpha_5^{state}$ & -0.04 & 0.00 & 0.27 & -0.67 & -0.14 & -0.01 & 0.07 & 0.49 & 4508.57 & 1.00 \\ 
  $\alpha_6^{state}$ & 0.00 & 0.00 & 0.29 & -0.61 & -0.11 & 0.00 & 0.11 & 0.64 & 14340.61 & 1.00 \\ 
  $\alpha_7^{state}$ & 0.00 & 0.00 & 0.30 & -0.65 & -0.11 & 0.00 & 0.11 & 0.66 & 13683.76 & 1.00 \\ 
  $\alpha_8^{state}$ & -0.02 & 0.00 & 0.24 & -0.56 & -0.12 & -0.00 & 0.09 & 0.50 & 6369.54 & 1.00 \\ 
  $\alpha_9^{state}$ & -0.00 & 0.00 & 0.30 & -0.65 & -0.11 & -0.00 & 0.11 & 0.64 & 11822.21 & 1.00 \\ 
  $\alpha_{10}^{state}$ & -0.00 & 0.00 & 0.30 & -0.67 & -0.11 & -0.00 & 0.10 & 0.64 & 13655.71 & 1.00 \\ 
  $\alpha_{11}^{state}$ & -0.00 & 0.00 & 0.31 & -0.67 & -0.12 & -0.00 & 0.11 & 0.66 & 12288.08 & 1.00 \\ 
  $\alpha_{12}^{state}$ & -0.00 & 0.00 & 0.30 & -0.66 & -0.11 & -0.00 & 0.11 & 0.63 & 10459.35 & 1.00 \\ 
  $\alpha_{13}^{state}$ & 0.00 & 0.00 & 0.29 & -0.65 & -0.11 & -0.00 & 0.11 & 0.63 & 11717.81 & 1.00 \\
  $\alpha_{14}^{state}$ & 0.00 & 0.00 & 0.30 & -0.65 & -0.11 & -0.00 & 0.11 & 0.65 & 11918.50 & 1.00 \\
  $\alpha_{15}^{state}$ & -0.00 & 0.00 & 0.26 & -0.56 & -0.11 & 0.00 & 0.10 & 0.57 & 7866.88 & 1.00 \\ 
  $\alpha_{16}^{state}$ & -0.00 & 0.00 & 0.31 & -0.68 & -0.11 & -0.00 & 0.11 & 0.66 & 12413.47 & 1.00 \\ 
  $\alpha_{17}^{state}$ & -0.00 & 0.00 & 0.30 & -0.66 & -0.11 & -0.00 & 0.11 & 0.66 & 12689.82 & 1.00 \\ 
  $\alpha_{18}^{state}$ & -0.00 & 0.00 & 0.30 & -0.68 & -0.11 & 0.00 & 0.11 & 0.66 & 15471.48 & 1.00 \\
  $\alpha_{19}^{state}$ & 0.00 & 0.00 & 0.29 & -0.64 & -0.10 & 0.00 & 0.11 & 0.65 & 12943.23 & 1.00 \\ 
  $\alpha_{20}^{state}$ & -0.03 & 0.00 & 0.25 & -0.61 & -0.14 & -0.01 & 0.07 & 0.48 & 4880.30 & 1.01 \\
  $\alpha_{21}^{state}$ & 0.00 & 0.00 & 0.26 & -0.56 & -0.10 & -0.00 & 0.10 & 0.57 & 6802.89 & 1.00 \\ 
  $\alpha_{22}^{state}$ & 0.00 & 0.00 & 0.29 & -0.64 & -0.11 & 0.00 & 0.11 & 0.64 & 14606.26 & 1.00 \\ 
  $\alpha_{23}^{state}$ & -0.00 & 0.00 & 0.30 & -0.66 & -0.11 & -0.00 & 0.11 & 0.66 & 12520.77 & 1.00 \\ 
  $\alpha_{24}^{state}$ & -0.00 & 0.00 & 0.30 & -0.66 & -0.11 & -0.00 & 0.10 & 0.66 & 11048.15 & 1.00 \\ 
  $\alpha_{25}^{state}$ & -0.00 & 0.00 & 0.30 & -0.64 & -0.11 & -0.00 & 0.11 & 0.65 & 10151.39 & 1.00 \\ 
  $\alpha_{26}^{state}$ & -0.00 & 0.00 & 0.30 & -0.66 & -0.12 & -0.00 & 0.11 & 0.64 & 12545.84 & 1.00 \\ 
  $\alpha_{27}^{state}$ & 0.00 & 0.00 & 0.29 & -0.64 & -0.11 & 0.00 & 0.11 & 0.64 & 14688.05 & 1.00 \\ 
  $\alpha_{28}^{state}$ & -0.00 & 0.00 & 0.29 & -0.65 & -0.11 & 0.00 & 0.11 & 0.63 & 13422.73 & 1.00 \\
  $\alpha_{29}^{state}$ & -0.00 & 0.00 & 0.30 & -0.64 & -0.11 & -0.00 & 0.10 & 0.65 & 11528.96 & 1.00 \\ 
  $\alpha_{30}^{state}$ & -0.00 & 0.00 & 0.30 & -0.67 & -0.11 & -0.00 & 0.11 & 0.65 & 13951.41 & 1.00 \\ 
  $\alpha_{31}^{state}$ & 0.00 & 0.00 & 0.30 & -0.67 & -0.11 & 0.00 & 0.11 & 0.65 & 12860.85 & 1.00 \\ 
  $\alpha_{32}^{state}$ & -0.00 & 0.00 & 0.26 & -0.54 & -0.11 & -0.00 & 0.10 & 0.56 & 6629.28 & 1.00 \\
  $\alpha_{33}^{state}$ & -0.00 & 0.00 & 0.29 & -0.65 & -0.11 & -0.00 & 0.11 & 0.62 & 13174.63 & 1.00 \\ 
  $\alpha_{34}^{state}$ & 0.00 & 0.00 & 0.29 & -0.64 & -0.11 & -0.00 & 0.11 & 0.64 & 13039.06 & 1.00 \\
  $\alpha_{35}^{state}$ & 0.00 & 0.00 & 0.30 & -0.67 & -0.11 & -0.00 & 0.11 & 0.65 & 12938.33 & 1.00 \\
  $\alpha_{36}^{state}$ & 0.00 & 0.00 & 0.29 & -0.63 & -0.11 & 0.00 & 0.11 & 0.62 & 14181.21 & 1.00 \\ 
  $\alpha_{37}^{state}$ & -0.00 & 0.00 & 0.29 & -0.64 & -0.11 & 0.00 & 0.10 & 0.64 & 11528.17 & 1.00 \\
  $\alpha_{38}^{state}$ & 0.00 & 0.00 & 0.30 & -0.64 & -0.10 & -0.00 & 0.11 & 0.67 & 13059.82 & 1.00 \\
  $\alpha_{39}^{state}$ & -0.00 & 0.00 & 0.30 & -0.65 & -0.11 & -0.00 & 0.10 & 0.65 & 13000.12 & 1.00 \\ 
  $\alpha_{40}^{state}$ & 0.01 & 0.00 & 0.30 & -0.63 & -0.11 & 0.00 & 0.11 & 0.65 & 12361.29 & 1.00 \\ 
  $\alpha_{41}^{state}$ & 0.00 & 0.00 & 0.29 & -0.63 & -0.10 & 0.00 & 0.11 & 0.67 & 13882.77 & 1.00 \\ 
  $\alpha_{42}^{state}$ & -0.00 & 0.00 & 0.31 & -0.70 & -0.11 & 0.00 & 0.11 & 0.66 & 11855.58 & 1.00 \\
  $\alpha_{43}^{state}$ & 0.00 & 0.00 & 0.31 & -0.65 & -0.11 & 0.00 & 0.11 & 0.68 & 13618.33 & 1.00 \\ 
  $\alpha_{44}^{state}$ & -0.00 & 0.00 & 0.30 & -0.66 & -0.11 & -0.00 & 0.11 & 0.63 & 11607.65 & 1.00 \\ 
  $\alpha_{45}^{state}$ & 0.00 & 0.00 & 0.31 & -0.66 & -0.11 & 0.00 & 0.11 & 0.69 & 9386.17 & 1.00 \\ 
  $\alpha_{46}^{state}$ & 0.00 & 0.00 & 0.30 & -0.65 & -0.11 & 0.00 & 0.11 & 0.66 & 12808.94 & 1.00 \\ 
  $\alpha_{47}^{state}$ & 0.02 & 0.00 & 0.26 & -0.54 & -0.09 & 0.01 & 0.12 & 0.62 & 8309.47 & 1.00 \\ 
  $\alpha_{48}^{state}$ & 0.00 & 0.00 & 0.30 & -0.64 & -0.11 & 0.00 & 0.11 & 0.66 & 12433.56 & 1.00 \\ 
  $\alpha_{49}^{state}$ & 0.00 & 0.00 & 0.30 & -0.63 & -0.11 & 0.00 & 0.12 & 0.67 & 13553.05 & 1.00 \\ 
  $\alpha_{50}^{state}$ & -0.00 & 0.00 & 0.29 & -0.65 & -0.11 & -0.00 & 0.11 & 0.63 & 11518.66 & 1.00 \\ 
  $\alpha_{51}^{state}$ & 0.08 & 0.01 & 0.27 & -0.39 & -0.04 & 0.03 & 0.18 & 0.76 & 1068.15 & 1.01 \\ 
  $\alpha_{r(1),1}^{region}$ & 0.48 & 0.01 & 0.32 & -0.06 & 0.23 & 0.47 & 0.71 & 1.09 & 563.06 & 1.01 \\ 
  $\alpha_{r(1),2}^{region}$ & 0.32 & 0.01 & 0.57 & -0.63 & -0.04 & 0.21 & 0.63 & 1.67 & 2569.76 & 1.00 \\ 
  $\alpha_{r(2),1}^{region}$ & 0.46 & 0.01 & 0.31 & -0.06 & 0.22 & 0.46 & 0.68 & 1.06 & 644.11 & 1.01 \\ 
  $\alpha_{r(2),2}^{region}$ & 0.41 & 0.01 & 0.53 & -0.40 & 0.03 & 0.31 & 0.71 & 1.68 & 1578.49 & 1.00 \\ 
  $\alpha_{r(3),1}^{region}$ & 0.39 & 0.01 & 0.34 & -0.15 & 0.11 & 0.36 & 0.63 & 1.10 & 642.92 & 1.01 \\ 
  $\alpha_{r(3),2}^{region}$] & -0.02 & 0.01 & 0.54 & -1.14 & -0.30 & -0.01 & 0.25 & 1.13 & 8663.30 & 1.00 \\ 
  $\alpha_{r(4),1}^{region}$ & 0.05 & 0.01 & 0.28 & -0.44 & -0.14 & 0.01 & 0.23 & 0.67 & 1127.36 & 1.01 \\ 
  $\alpha_{r(4),2}^{region}$ & 0.27 & 0.01 & 0.56 & -0.63 & -0.08 & 0.15 & 0.56 & 1.63 & 2519.12 & 1.00 \\ 
  $\alpha_{r(5),1}^{region}$ & 0.42 & 0.01 & 0.28 & -0.04 & 0.21 & 0.41 & 0.61 & 0.98 & 761.66 & 1.01 \\ 
  $\alpha_{r(5),2}^{region}$ & 0.78 & 0.02 & 0.66 & -0.13 & 0.24 & 0.68 & 1.20 & 2.31 & 740.39 & 1.01 \\ 
  $\alpha_{r(6),1}^{region}$ & 0.58 & 0.01 & 0.35 & -0.02 & 0.31 & 0.58 & 0.83 & 1.26 & 560.77 & 1.01 \\ 
  $\alpha_{r(6),2}^{region}$ & 0.33 & 0.01 & 0.53 & -0.53 & -0.02 & 0.23 & 0.62 & 1.58 & 2917.65 & 1.00 \\ 
  $\alpha_{r(7),1}^{region}$ & 0.49 & 0.01 & 0.31 & -0.04 & 0.25 & 0.49 & 0.71 & 1.09 & 595.74 & 1.01 \\ 
  $\alpha_{r(7),2}^{region}$ & 0.30 & 0.01 & 0.50 & -0.49 & -0.02 & 0.21 & 0.58 & 1.46 & 2151.45 & 1.00 \\ 
  $\alpha_{r(8),1}^{region}$ & 0.32 & 0.01 & 0.28 & -0.16 & 0.10 & 0.31 & 0.52 & 0.91 & 718.72 & 1.01 \\ 
  $\alpha_{r(8),2}^{region}$ & 0.52 & 0.02 & 0.64 & -0.36 & 0.05 & 0.38 & 0.87 & 2.09 & 1158.01 & 1.00 \\ 
  $\mu$ & 2.59 & 0.02 & 0.40 & 1.98 & 2.28 & 2.55 & 2.85 & 3.44 & 496.29 & 1.01 \\ 
  $c$ & -0.49 & 0.00 & 0.14 & -0.80 & -0.58 & -0.48 & -0.40 & -0.25 & 1026.26 & 1.01 \\ 
  $p$ & 0.69 & 0.01 & 0.19 & 0.35 & 0.56 & 0.68 & 0.81 & 1.07 & 1021.97 & 1.00 \\ 
   \hline
\end{tabular}
\caption{\textbf{Summary of posterior distribution from model fit}. Each row is a coefficient defined in equation 2 (main text) with terminology as described in the text. Each column is a descriptor of the posterior distribution of that coefficient across all five chains, specifically: its mean, standard error, standard deviation, 2.5$^{th}$, \nth{25}, \nth{50} (median), \nth{75}, 97.5$^{th}$, number of effective samples, and the $\hat{r}$ estimator of chain convergence\cite{Gelman1992}.}
\label{tab:posterior_summary}
\end{table}



\clearpage

\section{Contribution of Bayesian model terms to $R_t$}

Here we provide a visualisation showing the influence of our various model terms on $R_t$ for each state through time (fig \ref{fig:multiplot}). When reductions in mobility are strong, $R_t$ is largely driven by the mobility terms in the model, however when mobility reductions are weak, temperature and population density are much greater drivers of $R_t$. See for example New York: when mobility was close to baseline in February and March, fluctuating temperatures cause large spikes in $R_t$, which is also generally driven upwards by the relatively high population density of the state. However, after stay-at-home measures were implemented, $R_t$ drops and closely tracks changes in mobility, with temperature fluctuations having much lesser an effect. This shows that although we don't explicitly model an interaction between environment and mobility, our model represents this effect. 


\begin{figure}[ht]
  \centering
  \includegraphics[width=.9\linewidth]{figures/multiplot_figure.pdf}  
  \caption{ {\bf Relative contribution of model terms driving changes in $R_t$ through time.} The black dotted line shows the ``basic \RO'' term, which is constant through time for all states. The dashed lines represent temperature (blue), population density (red), combined mobility reductions (orange) and the autoregressive term (purple) and show what $R_t$ would be estimated at if the model only considered each of these terms alone. The grey soild line shows the realised $R_t$ estimated by the model. Essentially this shows the relative contribution of the model terms in pulling $R_t$ away from the baseline \RO (black dotted) to the realised $R_t$ (grey soild).}
\label{fig:multiplot}
\end{figure}

\clearpage
\bibliographystyle{unsrtnat}
\bibliography{library_dupfree}

\end{document}
%%% Local Variables:
%%% mode: latex
%%% TeX-master: t
%%% End:
